\section{Introduction}
\label{sec:intro}

As reported recently, the energy consumption of data centers keeps growing while more and more enterprises and organizations are building their own data centers\cite{urgaonkar2011optimal,Koomey2011}. The increasing speed of the data center energy consumption is approximately 10-12\% per year recently \cite{ghatikar2014demand}. The carbon emission and environment pollution issues attract insights of seeking for clean energy resources. Many IT companies such as Google\cite{GoogleGreen}, Apple\cite{Apple13}, and McGrawHill\cite{McGrawHill11} are trying to build their data centers together with renewable energy power plants.

Usually, generation of sustainable energy like solar and wind will be closely related to the weather condition at certain locations. As far as we know, there might be different choices for building the data center and the power plants. Both on-site and off-site generation are possible approaches \cite {Goiri13}. By grid-centric approaches, the renewable energy will be generated at locations with sufficient renewable sources and pumped into the grid. On the other hand, co-location and self-generation approaches sit the data center and the power plant at the same location to facilitate management and avoid long-distance losses. Since it seems no approach is perfect, we argue that companies essentially expect benefits from the investments by building up data centers and green power plants. However, since the renewable energy generation is mostly intermittent and sometimes might bring great penetration current into the electricity grid, the capacity should be carefully planned and will be limited from the perspective of grid operators. For example, some tiny failures may make an area of grid system completely out of power \cite{nytimes2014}.

Since data centers are becoming quite large loads for the electricity grid, they are supposed to have a significant impact on the operation of power grid \cite{haowang2014grid}. Large data center loads might increase the grid load and also lead to significant load variability of the electricity system. Especially, the emergence of the renewable energy sources with intermittent nature brings fluctuations onto the electricity networks. Furthermore, grid losses are on of the largest expenses for the power system operators \cite{de2014investigation}. EIA (U.S. Energy Information Administration) \cite{EIA} has estimated that the electricity transmission and distribution losses all over the U.S. is about 6\% of the electricity that is transmitted and distributed each year (averaged from 1990 to 2012). Thus, reduction of these losses for the grid can greatly affect the total operational costs.

In this case, companies who want to build data centers with either on-site or off-site renewable energy plants have to get the permission from the grid operators first in order to reduce the unexpected influence to the grid operation. Since the grid losses will also finally turn into expenses for end-users, service providers may want to collaborate with grid operators to minimize the overall cost when planning locations and capacity for the data centers. In the current literature, data center placement issues have been mentioned in some prior work\cite{Goiri11place,Dalger05,Boley09,larumbe2012optimal}, and there are also some research focused on the capacity planning of green data centers \cite{Le10,berral2014building}. Nevertheless, these work hasn't considered the impact of data center placement on the grid itself, which might also lead to comparable costs as other costs for data centers.

In this paper, we attempt to set up a different point of view, by combining the consideration for cloud service providers and energy companies together and aiming at the minimization of the overall cost for both. First, we investigate the impact of data center placement and its importance to the grid by studying a region of grid network. We pay special attention to the data center size, data center locations and the variation of renewable power generations. Second, we formulate the optimization framework, which incorporates the costs of data centers, renewable power plants and power grid into one objective. We also try to solve it under necessary constraints by using several different approaches. Then we conduct a case study in the New England area of the United States, by sitting and provisioning data centers and power plants at different locations, with the purpose of minimizing the overall cost. Results show that grid losses can have remarkable impact on the decision of selecting best locations for green data centers. Furthermore, the co-location choices by sitting the data center and green plants together don't show advantages despite of the reduced line cost and distribution cost, which is not that intuitive as prior work thought.

\myparagraph{Contributions of this paper.} The main contributions of this paper includes: (i) it quantifies the potential impact of data center placement on the bus network of power grid, and (ii) it proposes a framework for smart placement of both data centers and renewable energy plants in the power grid network. (iii) it formulates an optimization problem and gives a solution approach to find out good choices for sitting and provisioning data centers when considering grid costs.

To the extent of our knowledge, there is no previous work considering the jointly placement issues of data centers and green power plants while caring about the grid operational costs together. The remainder of the paper is organized as follows. Section \ref{sec:quantify} first quantifies the potential of data centers together with wind farms by placing them into different buses of the grid network system. In Section \ref{sec:framework}, we describe the optimization framework in detail, showing the integration of various parameters of the entire problem. Section \ref{sec:eval} evaluates the costs and illustrates breakdown of the total cost by different kinds of strategies. In Section \ref{sec:related}, we present some prior work related to this paper. Finally, the conclusion is given in Section \ref{sec:conclusion}.


%We discuss each of these in detail as following:(1) \textbf{The potential of data center placement on grid.} (2) \textbf{Jointly placement of data centers and renewable energy power plants.}


\section{Cost-based Placement}
\label{sec:framework}

As shown in the last section, the placement of new datacenters and wind farms can have significant impact on a grid transmission system.  Thus, in this section, we develop a framework for finding the best possible placement.  For simplicity, we assume that one datacenter and one wind farm are being placed in the system simultaneously.  Our framework can be easily extended to account for the simultaneous placement of multiple datacenters and multiple wind farms; of course, solving the optimization problem will become much more challenging for such cases.  Our optimization seek to find locations for the datacenter and wind farm that lead to the lowest total cost for construction and operation.  The overall framework is based on computing the cost throughout a year, using amortized capital costs and operational costs for the datacenter and operational revenue of the wind farm, along with system loss in the transmission network.

% By grid-aware placement, we attempt to efficiently select a set of locations for one or more datacenters to support a given amount of computational power, as well as one or more green power plants (e.g. solar, wind or others) to provide a given power capacity. The main goal is to minimize the overall cost for datacenters, green power plants and also the grid network operation. The next subsections defines some important parameters in the selection procedure. Then, the cost model and the entire optimization problem will be formulated and described.

\begin{table}[ht]
\caption{Framework parameters.  $l$ is a location, and $t$ is a time period.}
\begin{center}
\begin{tabular}{|l|p{1.9in}|r|}
\hline
\textbf{Symbol} & \textbf{Meaning} & \textbf{Unit}\\
\hline
$dcCapacity$ & desired power capacity for computing in DC & kW \\
$wfCapacity$ & desired power production capacity of wind farm & kW \\
\hline \hline
$pLand(l)$ & land price at $l$ & \$/m$^2$ \\
\hline \hline
$PUE(l,t)$ & PUE at $l$ during $t$ & \\
$maxPUE(l)$ & maximum PUE at $l$ & \\
$dcArea$ & land needed per kW of DC compute capacity &  m$^2$/kW \\
$cLinePow(l)$ & cost to layout power line from $l$ to the closest power plant & \$ \\
$cLineNet(l)$ & cost to layout optical fiber from $l$ to closest network backbone & \$ \\
$pBuildDC(c)$ & per kW price of building a datacenter with $c$ power capacity & \$/kW \\
$serverPow$ & server peak power demand & kW/serv \\
$switchPow$ & switch peak power demand & kW/switch \\
$servsSwitch$ & number of servers per switch & servs/switch \\
$pServer$ & price of a server &  \$/serv \\
$pSwitch$ & price of a network switch & \$/switch \\
$pNBWServ$ & cost of external network bandwidth per server & \$/serv-month\\
$pEnergy(l)$ & grid electricity price at $l$ & \$/kWh \\
$powNeed(t)$ & avg computing power demand of DC during $t$ &  kW \\
\hline \hline
$\beta(l,t)$ & avg generation efficiency of wind energy at $l$ during $t$ &  \%  \\
$wfArea$ & land needed per kW wind power & \$/m$^2$ \\
$pBuildWF$ & per kW price of building a wind power plant & \$/kW \\
%$avgEff(l)$ & avg generation efficiency of wind energy at $l$ & \% \\
$revEnergy(l)$ & revenue for selling wind energy to grid at $l$ & \$/kWh \\
\hline \hline
$transLoss(t)$  & avg system transmission loss in grid during $t$ & kW \\
%$disLoss(t)$ & the losses for distributing power to datacenter during time epoch $t$ & kW \\
$pTransLoss$ & the price for system transmission losses per kWh & \$/kWh \\
%$numLineVio(t)$  & the number of line capacity violation during time epoch $t$   &  \#  \\
%$numVolVio(t)$  & the number of voltage level violation during time epoch $t$  &  \#  \\
$V_i$ & the voltage level of the $i$th bus & p.u. \\
$V^{min}_i$ & the minimum accepted voltage of the $i$th bus &p.u.\\
$V^{max}_i$ & the maximum accepted voltage of the $i$th bus & p.u. \\
$P_k$ & the current line flow on the $k$th branch & MW \\
$P^{max}_k$ & the maximum allowed line flow of the $k$th branch & MW \\
\hline
\end{tabular}
\label{tab:par_setting}
\end{center}
\end{table}

\subsection{Optimization Framework}

Table \ref{tab:par_setting} lists the set of parameters in our framework.  Using these parameters, we define the optimization problem shown in Figure~\ref{fig:optimization}.  The objective of this optimization problem is to minimize the total cost ($totalCost$) of building and operating a datacenter of a given size ($dcCapacity$) and a wind farm of a given size ($wfCapacity$). %\xynote{In the program, wind farm size is not given but calculated based on the datacenter size and other location-dependent parameters.}
The datacenter and wind farm can each be placed at any location within a set of given locations.  The total cost has three components, the cost of the datacenter ($dcCost$), the cost of the wind farm ($wfCost$), and the cost of losses in the transmission system ($transCost$).

\begin{figure*}
Minimize $totalCost$ subject to
\begin{eqnarray}
V^{min}_i  \leq  V_i \leq V^{max}_i\\
P_k \leq P^{max}_k
\end{eqnarray}
where
\begin{eqnarray}
 	totalCost & = & dcCost + wfCost + transCost \\
	dcCost & = & dcCAPEX + dcOPEX \\
        dcCAPEX & = & dcLandCost + dcBuildCost + dcITCost \\
        dcOPEX & = & dcNetCost + dcEnergyCost \\
        dcLandCost & = & pLand(d) \cdot dcArea \cdot dcCapacity \\
        dcBuildCost & = & dcTotalPow \cdot pBuildDC(dcTotalPow) +
            cLinePow(d) + cLineNet(d) \\
        dcTotalPow & = & dcCapacity \cdot maxPUE(d) \\
        dcITCost & = & nServers \cdot pServer + nSwitches \cdot
            pSwitch \\
        nServers & = & dcCapacity / (serverPow + switchPow / servsSwitch)\\
        nSwitches & = & nServers / servsSwitch\\
        dcNetCost & = & nServers \cdot pNBWServ \\
        dcEnergyCost & = & \sum_{t \in T} {|t| \cdot powNeed(t) \cdot PUE(d,t) \cdot pEnergy(d) } \\
 	wfCost & = & wfCAPEX - wfRev  \\
        wfCAPEX & = & wfLandCost + wfBuildCost \\
        wfLandCost & = & pLand(w) \cdot wfArea \cdot wfCapacity \\
        wfBuildCost & = & pBuildWF \cdot wfCapicity + cLinePow(w) \\
        wfRev & = & revEnergy(w) \cdot  \sum_{t \in T}{ |t| \cdot
            \beta(w,t) \cdot wfCapicity } \\
        transCost & = & pTransLoss \cdot \sum_{t \in T}{ |t| \cdot transLoss(t)} % + disLoss(t))}\\
\end{eqnarray}
\caption{Optimization framework.  The datacenter is placed at location $d$ and the windfarm is placed at location $w$.  The objective is to minimize $totalCost$ for a given time period $T$ (divided into epochs denoted by $t$) and a set of possible locations for $d$ and $w$.  $|t|$ denotes the length of epoch $t$.}
\label{fig:optimization}
\end{figure*}


\subsubsection{Datacenter} The cost of the datacenter can be broken down into capital ($dcCAPEX$) and operational ($dcOPEX$) components.  The capital costs are those investments made upfront and depreciated over the lifetime of the datacenter.  These costs include the cost for buying land ($dcLandCost$), building the datacenter ($dcBuildCost$), and buying IT equipment ($dcITCost$).  The cost of building the datacenter include datacenter construction cost as well as the costs of laying power and network lines to the datacenter.  IT equipment includes servers and switches.  Land price varies according to location ($pLand(d)$ for location $d$), whereas the other prices do not to a first approximation.  Of course, the total cost of laying the power ($cLinePow(d)$) and network ($cLineNet(d)$) lines depends on location, as the distances to the closest power plant and network backbone are location-dependent.  The datacenter construction cost is typically estimated as a function of the maximum power to be consumed by the datacenter.  This maximum power is that required by the maximum number of servers and network switches when running at 100\% utilization times the maximum expected PUE of the datacenter.  The PUE is computed by dividing the overall power consumption by the power consumption of the computational equipment.  The PUE is higher when temperature and/or humidity are high, since cooling consumes more energy under those conditions.

The operational costs are those incurred during the operation of the datacenter, and include costs for external network bandwidth use ($dcNetCost$) and the grid electricity ($dcEnergyCost$) required to run the datacenter.  (There is also a cost for water, which is currently not considered by can be easily added.)  The electricity cost is computed based on the IT equipment's power demand over time ($powNeed(t)$), the PUE, and the electricity price.  Both the electricity price and the PUE vary with location.

Finally, lower taxes and one-time incentives are another important component of the cost of a datacenter.  For example, some states in the US lower taxes on datacenters, as they generate employment and wealth around them.  This component depends on the nature of the savings and applies to each cost in a different way.  Although we do not consider this component further, it is easy to add it to our framework.

\subsubsection{Wind farm}  The cost of the wind farm is modeled as the capital cost ($wfCAPEX$) minus the revenue earned by selling the wind energy to the grid ($wfRev$).  We assume that the operational cost of operating the wind farm is low, and so do not consider it here; of course, this cost can be easily added to the framework.  The capital costs include the cost for buying land ($wfLandCost$) and building the wind farm ($wfBuildCost$), which in turn includes the construction cost and the cost of laying the power line from the wind farm to the closest power plant.  The construction cost is assumed to be a linear function of the desired power generation capacity.  Note that if the datacenter and wind farm are co-located, then the cost for laying power lines is incurred only once.

The revenue earned by the wind farm is computed over the time period $T$, where the energy generated within any time epoch $t$ in $T$ at location $w$ depends on the efficiency of the wind turbines and the wind speed.  The efficiency of today's wind turbine is close to 50\%.  We capture the efficiency and impact of wind speed in epoch $t$ using the parameter $\beta(w,t)$, which gives the fraction of the wind farm's maximum capacity actually produced during $t$.

\subsubsection{Transmission system} As discussed above, adding a datacenter and wind farm to an existing transmission system will alter the power flow of the network, thus affecting the system transmission loss.  We model this loss across the entire time period $T$, and assume that each unit of loss has a corresponding cost.

% On the other hand, during the transmission process the line capacity might be violated if the power flow exceeds the limit. We use $numLineVio(t)$ and $numVolVio(t)$ to denote the number of line capacity violations and voltage violations during power transmission. The grid operator will try to avoid such violations when operating the power grid system.


\subsubsection{Constraints}

The major constraints that must be observed for the above optimization is that there must not be any voltage and transmission line capacity violations.

\subsection{Solving the Optimization Problem}

Currently, we perform a complete search over all possible placement of the datacenter and wind farm to find the best solution to the optimization problem.  This approach works well for the scale that we are studying (e.g., the New England system).  However, as the system scale to much larger sizes, a more scalable approach would eventually be needed.  We leave this for future work.  (Note that the New England system that we study already covers a large area -- most of the North Eastern region of the U.S. and some parts of Canada -- so that it may not be necessary to study much larger systems.)

% datacenters and renewable power plant, as well as the capacity provisioned at each location for datacenters or green power plants (if any).

% The overall cost should be optimized under the constraints, which are listed in Figure~\ref{fig:constraints}. Equation 21-23 show the constraints of the provisioned capacity for datacenters.

% Equation 24 means that the provisioned capacity for renewable energy plants are determined by the total power demand from the datacenters. This constraint is added indicating that the power generation and consumption added to the grid should be balanced from the perspective of the grid system.

% Furthermore, Equation 26
% is a strict limitation for keeping the grid out of any violations at any given time $t$, since we assume the grid reliability is crucial and must be guaranteed.

% \begin{figure*} [ht]
% \begin{small}
% \centering
% \begin{eqnarray}
% \forall_{l \in \mathcal{L}}, capDC(l) \leq DC(l) \cdot CapacityDC
% &\Rightarrow& \text{capacity of datacenter at $l$ should be zero when $DC(l)$ is 0} \\
% \sum_{l\in \mathcal{L}}{DC(l)\cdot capDC(l)} = CapacityDC
% &\Rightarrow& \text{total capacity of built datacenters should meet the requirement} \\
% \forall_{t \in T}, demand(l,t) \leq capDC(l)
% &\Rightarrow& \text{power demand of the datacenter should not exceed its capacity} \\
% %\forall_{l \in \mathcal{L},r\in \mathcal{R}}, capRE(l,r) \leq RE(l,r) \cdot CapacityRE  & \Rightarrow &
% %\text{capacity of type $r$ plant at $l$ should be zero if $RE(l,r)$ is 0} \\
% %\sum_{l \in \mathcal{L},r \in \mathcal{R}}{RE(l,r) \cdot capRE(l,r)} = CapacityRE & \Rightarrow &
% %\text{total capacity of power plant should meet the requirement}\\
% \begin{split}
% \sum_{l \in \mathcal{L},r \in \mathcal{R}}{ RE(l,r) \cdot capRE(l,r) \cdot avg\textit{Eff}(l,r) } = \\
% \sum_{t \in T, l\in \mathcal{L}}{DC(l) \cdot demand(l,t)\cdot PUE(l,t)}
% \end{split}
% &\Rightarrow &\text{the generated green energy should be balanced with consumption} \\
% \forall_{t \in T}, 0 \leq \textit{eff}RE(r,l,t) < 1
% &\Rightarrow&  \text{efficiency of power plant should be between $[0,1)$} \\
% \forall_{t \in T}, numLineVio(t)=0, numVolVio(t)=0
% &\Rightarrow& \text{No violations in each time epoch}
% \end{eqnarray}
% \end{small}
% \caption{Optimization constraints.}
% \label{fig:constraints}
% \end{figure*}

%%% Local Variables:
%%% mode: latex
%%% TeX-master: "paper"
%%% End:

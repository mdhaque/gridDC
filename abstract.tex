\begin{abstract}
  Datacenters are being constructed at a rapid pace.  Concurrently,
  offsetting renewable power plants are also being built to mitigate
  the environmental impact of the new datacenters' massive energy
  consumption.
%These renewable power plants may or may not be co-located with datacenters.
  Research efforts that have studied how to best place new datacenters
  and renewable power plants have mostly neglected the impact on the
  electricity transmission grid.
%With rapidly increasing penetration of such ``renewable powered datacenters,'' it is becoming imperative to study their impact on the transmission grid.
  In this paper, we show that accounting for the impact on the
  transmission grid
% when placing new datacenters and renewable power plants
can be mutually beneficial to both datacenter owners and grid
operators.  Specifically, locating datacenters and renewable power
plants at strategic places in the grid could help to minimize (i)
overloading of transmission lines, (ii) grid voltage variations
outside the acceptable range, and (iii) transmission system losses.
We develop an optimization framework for placing a new datacenter and
offsetting wind farm, and use it in a case study to show that
considering transmission losses along with datacenter costs
 can lead to different placements and
lower overall cost.  Interestingly, co-locating the datacenter and
wind farm does not always lead to lowest impact on the transmission
grid and lowest overall cost.  Thus, we conclude that the impact of
location on the costs of {\em both} the datacenter owners and grid
operators should be considered when placing new datacenters and
offsetting renewable power plants.
\end{abstract}


%%% Local Variables:
%%% mode: latex
%%% TeX-master: "paper"
%%% End:

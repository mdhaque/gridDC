\section{Related Work}
\label{sec:related}

As already mentioned, a number of previous efforts have studied the
placement of new datacenters.  Alger \cite{Dalger05} explained how to
choose an optimal location for a datacenter by considering hazards,
accessibility, and scalability factors.  Stansberr \cite{Stansberr06}
ranked some cities by estimating the annual operation costs of a
datacenter.  Oley \cite{Boley09} considered looking for a proper
location for a datacenter by investigating the
power rates of different states.  Goiri $\textit{et al.}$
\cite{Goiri11place} focused on intelligently finding the best places
for building multiple datacenters to form a network for interactive
Internet services.  Berral $\textit{et al.}$ \cite{berral2014building}
considered selecting sites for datacenters and on-site power plants
that support ``follow-the-renewables'' cloud services.  Gao $\textit{et al.}$
\cite{gao2013answer} studied how to site datacenters near existing
wind farms, and distributing load using a greedy online algorithm.
None of these works have considered the impact of placing new
datacenters on the transmission grid. 

A previous work that has considered the interaction of datacenters and
the grid is~\cite{liu2014pricing}.  In this work, Liu $\textit{et
  al.}$ show that adding renewable power plants (solar in their study)
can lead to voltage violations within a grid distribution system.
They also show that datacenters can help avoid such voltage
violations by dynamically adjusting their power demand based on
signals from the grid.  A number of research efforts have also studied
how datacenters can participate in demand response programs and
ancillary services to help ease the management of the
grid~\cite{Aikema12,AdamWierman2014}.  A key difference between these
works and ours is the assumption of datacenters being able to
dynamically adjust their power demands in response to grid signals.
In addition, they did not actually consider the placement of
datacenters, nor did they consider transmission system losses.

Mohsenian $\textit{et al.}$ in \cite{Mohsenian-Rad10grid} proposed a
request distribution policy among datacenters to ensure power load
balancing. They tried to minimize the maximum power on any
transmission line by distributing the computing requests to suitable
datacenters. Their work assumes that a fairly large number of
datacenters (e.g., 6) are connected to the same power distribution
network.  Further, they did not consider the impact of transmission system losses on the placement of datacenters.

% Larumbe $\textit{et al.}$ \cite{larumbe2012optimal} presented a mathematical problem aiming at solving the location and routing of cloud service components.
% Unlike our work, they didn't put insights the possible impact of datacenters and distributed energy generations on the utility grid system.
% A very close paper to our work is \cite{liu2014pricing} which considered a realistic distribution system and discussed how it impacts the voltage in the distribution system, while we are focusing on the transmission system of the grid.



%%% Local Variables:
%%% mode: latex
%%% TeX-master: "paper"
%%% End:
